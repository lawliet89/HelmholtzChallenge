\documentclass[11pt, oneside, a4paper]{article}
\usepackage[left=2.5cm, right=2.5cm, top=2.5cm]{geometry}
\usepackage{listings}
\usepackage{verbatim}
\usepackage{tabularx}
\usepackage[toc,page]{appendix}

\usepackage{titling}
\setlength{\droptitle}{-3cm}   % title offset

\setlength\parindent{0pt}
\setlength{\parskip}{4pt}

\date{}
\title{Helmholtz Challenge}
\author{
  \small{Oskar Weigl (ow610)}\\
  \small{Ryan Savitski (rs5010)}\\
  \small{Yong Wen Chua (ywc110)}
}

\begin{document}
\maketitle

\section{Introduction} % (fold)
\label{sec:introduction}

This report outlines the acceleration of the given FEM code using a GPU. The hardware architecture is a natural fit for the target problem, which exhibits significant data parallelism with latency tolerance towards individual elements of the iteration space. The application is compute-bound through floating point computations, which GPU architectures are designed to perform efficiently. Furthermore, there is no interleaving of IO or serial kernels, avoiding the need of copying any intermediate results between host and device.

% section introduction (end)

\section{Initial investigation} % (fold)
\label{sec:initial_investigation}

\subsection{Default implementation} % (fold)
\label{sub:default_implementation}

The computation is performed on triangular prisms with six edge nodes obtained from an extruded unstructured mesh. The vertical columns of nodes are a regular structure and therefore allow for the data to be laid out sequentially for each column, reducing the number of indirect memory accesses.

The given implementation consists of a serial execution of the following stages: 

 \texttt{expression 1} sets up the boundary conditions. Note that the autogenerated is particularly inefficient, initialising nodes by iterating over cells and computing the same result multiple times. This can easily be reimplemented as a serial loop over the coordinate array.

\texttt{lhs and rhs} kernels perform local assembly with eight double precision multiply-accumulate expressions per node of a cell (arranged in a reduction tree). Note that the two sides of the equation are independent and could be computed in parallel on a device with sufficient resources.

The provided implementation, processing the \texttt{small} mesh on a i7-3770K CPU (see appendix \ref{sub:cpu} for more details) takes (averaged over a set of runs) 72.8s (eval: 6.90s, interp: 2.58s, RHS: 13.4s, LHS: 46.3s). Dynamic voltage and frequency scaling (DVFS) was enabled, making the frequency of the relevant core float between 3.5 GHz and 3.9 GHz during execution.

% subsection default_implementation (end)

\subsection{Initial look at parallelisation} % (fold)
\label{sub:initial_look_at_parallelisation}

TODO

which loops looked like they had parallelisable iterations, which seemed to be most prospectful

initial look at how kernels can be parallelised and verifying correctness through tbb, which is a thingie thingie (library, no compiler dependencies)
and allows for quick parforing of loops


Given that the target device for final optimisation was an Nvidia GPU, with the intention to use CUDA language and tooling, the operating system was chosen to be Windows for profiling support. As a consequence, the timing code had to be ported, instead using the Windows API function `QueryPerformanceCounter` which provides time stamps with 1 microsecond resolution. The implementation is opaque, but guaranteed to be precise, likely using the invariant Time Stamp Counter (TSC) found on the CPU, with necessary OS-level compensation. Therefore the timing is accurate across threads, cores and independent of DVFS.


in particular, at this stage, the loops x and y in kernels z and w parfor'd, runtime were
runtimes with tbb, highlight amount of cores ran on (virtual, physical)


% subsection initial_look_at_parallelisation (end)

% section initial_investigation (end)

\section{GPU and CUDA} % (fold)
\label{sec:gpu_and_cuda}

TODO

\ref{sub:gpu} appendix of rough gpu details, might wish to include stuff on SM 3.0 and such

using cuda toolkit bla, compiler ver bla, hardware bla , GK104 Kepler

constant memory opt

hw for cos in expr1? 

mention no intrinsic double precision atomics -> implemented through CAS, probably not as efficient as it could be

register spilling, attempts to fix by transposing loops

in the end -> DP functional unit hazards being the true bottleneck
mention emulating DP through SP floats
mixed precision algorithms can be a decent tradeoff, literature exists on the subject
mention GK110 being specifically designed to improve DP performance over GK104.

inlining addto\_vector

TODO: How correctness was ensured... tester program?

one thread per cell -> adjacent threads in blocks are staked vertically

Checking disassembly, the constant matricies used in wrap\_lhs\_GPU were stored on the stack, in local memory. These were then accessed in the hot loop.
Moving constant data from the kernel to constant data memory improved execution time from 154ms to 64ms for 1.5M threads total (for 1/20th of the full small mesh, for testing)
From IPC 0.4 to 0.77.
Local memory from 133Mreq, 19GB to 61Mreq, 8GB.
In both cases we get 125GB/s, so we have a good indication that local memory may still be the bottleneck.
For comaprison, Global memory is currently 3.3Mreq/s.

Before transpose 70ms for 1/20, 116 after: transpose made it worse for some reason!

For RHS: 114ms before constant memory.

%%%
After basically only constant memory optimisations, we have total times:
CPU: 72.8s breakdown: (eval: 6.90, interp: 2.58, RHS: 13.4, LHS: 46.3)
GPU: 2.58s
%%%


Final GPU:
Total: 2.23 seconds. See timeline for breakdown. (27ms, 160ms, 715ms, 980ms)
Serialisation 57\%


% section gpu_and_cuda (end)

\section{Work partition} % (fold)
\label{sec:work_partition}
\begin{tabular}{ l l  }
Yong Wen Chua & - tbb, cuda. \\
Oskar Weigl & - cuda, profiling. \\
Ryan Savitski & - code analysis. \\
\end{tabular}
% section work_partition (end)

\clearpage
\begin{appendices}
\section{Summary and setup}

\subsection{CPU} % (fold)
\label{sub:cpu}

\begin{verbatim}
Processor 1     ID = 0
  Number of cores   4 (max 8)
  Number of threads 8 (max 16)
  Name      Intel Core i7 3770K
  Codename    Ivy Bridge
  Specification   Intel(R) Core(TM) i7-3770K CPU @ 3.50GHz
  Package (platform ID) Socket 1155 LGA (0x1)
  CPUID     6.A.9
  Extended CPUID    6.3A
  Core Stepping   E1/L1
  Technology    22 nm
  TDP Limit   77 Watts
  Tjmax     105.0 °C
  Core Speed    2807.4 MHz
  Multiplier x Bus Speed  28.0 x 100.3 MHz
  Stock frequency   3500 MHz
  Instructions sets MMX, SSE, SSE2, SSE3, SSSE3, SSE4.1, SSE4.2, EM64T, VT-x, AES, AVX
  L1 Data cache   4 x 32 KBytes, 8-way set associative, 64-byte line size
  L1 Instruction cache  4 x 32 KBytes, 8-way set associative, 64-byte line size
  L2 cache    4 x 256 KBytes, 8-way set associative, 64-byte line size
  L3 cache    8 MBytes, 16-way set associative, 64-byte line size
  FID/VID Control   yes

  Turbo Mode    supported, enabled
  Max non-turbo ratio 35x
  Max turbo ratio   39x
  Max efficiency ratio  16x
  Min Power   60 Watts
  O/C bins    unlimited
  Ratio 1 core    39x
  Ratio 2 cores   39x
  Ratio 3 cores   38x
  Ratio 4 cores   37x
  TSC       3510.2 MHz
  APERF     3730.9 MHz
  MPERF     3481.3 MHz
\end{verbatim}

\clearpage

\subsection{GPU} % (fold)
\label{sub:gpu}
\begin{verbatim}
Display adapter 0 
  Name      NVIDIA GeForce GTX 670
  Revision    A2
  Codename    GK104
  Technology    28 nm
  Memory size   2 GB
  PCI device    bus 1 (0x1), device 0 (0x0), function 0 (0x0)
  Vendor ID   0x10DE (0x3842)
  Model ID    0x1189 (0x2678)
  Performance Level 0
    Core clock  1006.0 MHz
    Memory clock  3105.0 MHz

Win32_VideoController   AdapterRAM = 0x80000000 (2147483648)
Win32_VideoController   DriverVersion = 9.18.13.3523
Win32_VideoController   DriverDate = 03/04/2014
\end{verbatim}

\subsection{Operating system} % (fold)
\label{sub:operating_system}
\begin{verbatim}
Windows Version     Microsoft Windows 7 (6.1)
                    Ultimate Edition 64-bit  Service Pack 1 (Build 7601) 
\end{verbatim}

\subsection{CUDA} % (fold)
\label{sub:cuda}
Compilation via:
\begin{verbatim}
nvcc -arch compute_30 -code sm_30 -O3
\end{verbatim}

\end{appendices}
%%%%%%%%%%%%%%%%%%%%%%%%%%%%%%%%%%%%%%%%%%%%%%%%%%%%%%%%%%%%%%%%%%%%%%%%%%%%%%%%%%%%%%%%%%%%%%%%%%%%%%%%%%%%%%%%%%%%%
%%%%%%%%%%%%%%%%%%%%%%%%%%%%%%%%%%%%%%%%%%%%%%%%%%%%%%%%%%%%%%%%%%%%%%%%%%%%%%%%%%%%%%%%%%%%%%%%%%%%%%%%%%%%%%%%%%%%%
%%%%%%%%%%%%%%%%%%%%%%%%%%%%%%%%%%%%%%%%%%%%%%%%%%%%%%%%%%%%%%%%%%%%%%%%%%%%%%%%%%%%%%%%%%%%%%%%%%%%%%%%%%%%%%%%%%%%%
%%%%%%%%%%%%%%%%%%%%%%%%%%%%%%%%%%%%%%%%%%%%%%%%%%%%%%%%%%%%%%%%%%%%%%%%%%%%%%%%%%%%%%%%%%%%%%%%%%%%%%%%%%%%%%%%%%%%%


\hspace{1em}
\hrule
general notes
\hspace{1em}
\hrule

\section{report notes \& deliverables} % (fold)
\label{sec:report_notes_on_deliverables}

Marks are awarded for:
\begin{itemize}
\item  Systematic analysis of the application's behaviour
\item  Systematic evaluation of performance improvement hypotheses
\item  Drawing conclusions from your experience
\item  A professional, well-presented report detailing the results of your work.
\end{itemize}

\subsection{} % (fold)
\label{sub:}

% subsection  (end)

What to hand in Hand in a concise report which
\begin{itemize}
\item  Explains what hardware and software you used,
\item  What hypothesis (or hypotheses) you investigated,
\item  How you evaluated what the potential advantage could be,
\item  How you explored the effectiveness of the approach experimentally
\item  What conclusions can you draw from your work
\item  Specify how you ensured the correct results were obtained or justify why that is not the case
\item  If you worked in a group, indicate who was responsible for what.
\end{itemize}

\subsection{} % (fold)
\label{sub:}

% subsection  (end)

Regarding the report for the Helmholtz challenge, to make marking easier in terms of the technical side of your experiments, and also to gain a clear understanding of what you did/tried I would like you to include a structured summary of the outcome of the various optimizations attempted.

Ideally this would be a table specifying:
\begin{itemize}
\item  the hardware you ran on including: CPU/GPU, processor model number, clock rate, number of cores, cache sizes
\item  the compiler you used: ICC, GCC, etc, the compiler version number
\item  the compiler flags -fopenmp -Ofast -avx  - or whatever your choice is
\item  the parallelization strategy/paradigm used, if any (threads, OpenMP,OpenCL, MPI, CUDA)
\item  the number or range of cores you ran on
\item  the execution times you measured, and the steps you took to ensure your results are reproducible (ie not subject to random variations)
\item  the speed-up you got. If it helps you can point to graphs in your report.
\end{itemize}
If you have tried several improvements to the code or you would like to show how you explored a certain range of optimizations then you can include intermediate results which show the effect of a particular optimization.

This table needn't be longer than half a page (a page at most if you really really have to) and should make the presentation of the results concise.  This can be outside your page budget.

% section report_notes_on_deliverables (end)

\end{document}


%%%%%%%%%%%%%%%%%%%%%%%%%%%%%%%%%%%%%%%%%%%%%%%%%%%%%%%%%%%%%%%%%%%%%%%%%%%%%%%%%%%%%%%%%%%%%%%%%%%%%%%%%%%%%%%%%%%%%
%%%%%%%%%%%%%%%%%%%%%%%%%%%%%%%%%%%%%%%%%%%%%%%%%%%%%%%%%%%%%%%%%%%%%%%%%%%%%%%%%%%%%%%%%%%%%%%%%%%%%%%%%%%%%%%%%%%%%
%%%%%%%%%%%%%%%%%%%%%%%%%%%%%%%%%%%%%%%%%%%%%%%%%%%%%%%%%%%%%%%%%%%%%%%%%%%%%%%%%%%%%%%%%%%%%%%%%%%%%%%%%%%%%%%%%%%%%
%%%%%%%%%%%%%%%%%%%%%%%%%%%%%%%%%%%%%%%%%%%%%%%%%%%%%%%%%%%%%%%%%%%%%%%%%%%%%%%%%%%%%%%%%%%%%%%%%%%%%%%%%%%%%%%%%%%%%
data parallel
tons of independent work
fp calculations


accel of cos-field through hw

const mem


register spilling in base cuda implementation
%%%%%%%%%%%%%%%%%%%%%%%%%%%%%%%%%%%%%%%%%%%%%%%%%%%%%%%%%%%%%%%%%%%%%%%%%%%%%%%%%%%%%%%%%%%%%%%%%%%%%%%%%%%%%%%%%%%%%
%%%%%%%%%%%%%%%%%%%%%%%%%%%%%%%%%%%%%%%%%%%%%%%%%%%%%%%%%%%%%%%%%%%%%%%%%%%%%%%%%%%%%%%%%%%%%%%%%%%%%%%%%%%%%%%%%%%%%
so: to write about in report (if you want to still focus on that): interchange does free up registers, reduces register spilling to the point of 0 if you do both i and j outside. However, most likely the compiler is hiding the serial latency of the massive serial sum of products statement using the iterations over i and j.
therefore, returning to the same accumulator quickly, you lose performance


[21:29:03] madcowswe: what is really interesting to write about is the ability for the compiler to do interleaving of different iterations to hide the latency (:
%%%%%%%%%%%%%%%%%%%%%%%%%%%%%%%%%%%%%%%%%%%%%%%%%%%%%%%%%%%%%%%%%%%%%%%%%%%%%%%%%%%%%%%%%%%%%%%%%%%%%%%%%%%%%%%%%%%%%
%%%%%%%%%%%%%%%%%%%%%%%%%%%%%%%%%%%%%%%%%%%%%%%%%%%%%%%%%%%%%%%%%%%%%%%%%%%%%%%%%%%%%%%%%%%%%%%%%%%%%%%%%%%%%%%%%%%%%
%%%%%%%%%%%%%%%%%%%%%%%%%%%%%%%%%%%%%%%%%%%%%%%%%%%%%%%%%%%%%%%%%%%%%%%%%%%%%%%%%%%%%%%%%%%%%%%%%%%%%%%%%%%%%%%%%%%%%


\section{General Approach}

The general approach taken to speed up the execution of the calculation is to exploit the parallelism inherent in the problem itself. In particular, by studying the source code, we can see that for each of the wrapper function calls, the cells calculations are mostly independent of each other. The cell calculations can therefore be made to run in parallel. Due to unstructured nature of the mesh, calculations over the entire space must be complete before the next call to the next wrapper function can be made.

In each of the cell, calculation is written to a buffer in which each element of the buffer corresponds to one vertex in the mesh. Since a vertex can belong to more than one cell, care must be taken to ensure that the values accumulated in each vertex is done so without any race conditions such as an atomic add.

The unstructured nature of the mesh also means that trying to exploit spatial locality might be difficult. It will also be difficult to divide the entire iteration space into independent blocks because the memory locations are not ``adjacent'' to each other. 

\section{Threading}

The first approach to speeding up the calculation was to use threads to exploit the parallelism inherent in the problem. Using the Threading Building Blocks library\footnote{https://www.threadingbuildingblocks.org/}, the iteration space was made into what is essentially a parallel for loop that the library can schedule to run in parallel using threads depending on the hardware available. 

There are two avenues for parallerise the work being done: over each ``column'' of cells or over each cell individually.  We hypothesise that while both variants of threading will bring about performance improvements to the computation over the original serialised version, the former parallel version is expected to work better than the latter. This is because the work done per cell might be too small compared to the overheads of thread scheduling and context switching. 

The results shown in table \ref{table:threads} confirms our hypothesis. We notice that threading by columns gives the largest amount of speed up over the original version, by approximately a factor of four. For threading by cell, the performance for most operations are improved significantly, though not as much as threading by columns. Finally, interpolating expressions became slower in this version because of the overheads in thread scheduling when each thread does too little work.

\section{CUDA Acceleration}

one thread per cell -> adjacent threads in blocks are staked vertically

Checking disassembly, the constant matricies used in wrap_lhs_GPU were stored on the stack, in local memory. These were then accessed in the hot loop.
Moving constant data from the kernel to constant data memory improved execution time from 154ms to 64ms for 1.5M threads total (for 1/20th of the full small mesh, for testing)
From IPC 0.4 to 0.77.
Local memory from 133Mreq, 19GB to 61Mreq, 8GB.
In both cases we get 125GB/s, so we have a good indication that local memory may still be the bottleneck.
For comaprison, Global memory is currently 3.3Mreq/s.

Before transpose 70ms for 1/20, 116 after: transpose made it worse for some reason!

For RHS: 114ms before constant memory.

After basically only constant memory optimisations, we have total times:
CPU: 72.8s breakdown: (eval: 6.90, interp: 2.58, RHS: 13.4, LHS: 46.3)
GPU: 2.58s

Final GPU:
Total: 2.23 seconds. See timeline for breakdown. (27ms, 160ms, 715ms, 980ms)
Serialisation 57\%


\section{Summary Table}
\subsection{Threading Performance}
Testing was done on a laptop with Intel Core i7-4700MQ with 4 cores running on Linux 13.10. The programs were compiled with GCC with optimisation set to level 3. The results are summarised in the table below, and the times are consistent across several runs with very minor variations. The times for the steps ``Set array to zero'' and ``Evaluating expression'' are not reported because they are trivial cases that has been optimised by the compiler. The timings were generated from the large mesh.

\begin{table}[h]
\begin{tabularx}{1.2\textwidth}{ |X|X|X|X|X| }
\cline{2-5}
                                          & Evaluating Exp & Interpolating Exp & Assembling RHS & Assembling LHS \\ \hline
\multicolumn{1}{|l|}{Original Version}    & 40.0307        & 4.86542           & 21.0748        & 37.747         \\ \hline
\multicolumn{1}{|l|}{Threading by Cell}   & 12.993         & 5.59316           & 11.0813        & 14.9041        \\ \hline
\multicolumn{1}{|l|}{Threading by Column} & 7.77536        & 1.19127           & 5.64956        & 9.59484        \\ \hline
\end{tabularx}

\caption{Timings in seconds for threaded version using the large data mesh.}
\label{table:threads}
\end{table}
\subsection{CUDA Performance}

\end{document}
